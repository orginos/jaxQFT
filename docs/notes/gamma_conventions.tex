\documentclass[11pt]{article}
\usepackage{amsmath,amssymb,mathtools}
\usepackage[a4paper,margin=1in]{geometry}
\title{Gamma-Matrix Conventions in jaxQFT}
\author{jaxQFT}
\date{\today}

\begin{document}
\maketitle

\section{Scope}
This note documents the Euclidean gamma-matrix representation used in
\texttt{jaxqft/fermions/gamma.py} and how it is used in Wilson-type operators.

\section{Euclidean Clifford Algebra}
The implementation uses Hermitian Euclidean gamma matrices satisfying
\begin{equation}
\{\gamma_\mu,\gamma_\nu\} = 2\delta_{\mu\nu} I.
\end{equation}
For even dimensions, the chirality matrix is
\begin{equation}
\gamma_5 = i^{d/2}\prod_{\mu=0}^{d-1}\gamma_\mu.
\end{equation}

\section{Base Pauli Matrices}
\begin{equation}
\sigma_1=
\begin{pmatrix}
0 & 1\\
1 & 0
\end{pmatrix},\quad
\sigma_2=
\begin{pmatrix}
0 & -i\\
i & 0
\end{pmatrix},\quad
\sigma_3=
\begin{pmatrix}
1 & 0\\
0 & -1
\end{pmatrix}.
\end{equation}

\section{2D Representation}
For $d=2$, jaxQFT uses
\begin{equation}
\gamma_0=\sigma_1,\qquad
\gamma_1=\sigma_2.
\end{equation}

\section{4D Representation}
For $d=4$, jaxQFT uses the tensor-product representation
\begin{equation}
\gamma_0=\sigma_1\otimes\sigma_1,\quad
\gamma_1=\sigma_1\otimes\sigma_2,\quad
\gamma_2=\sigma_2\otimes I_2,\quad
\gamma_3=\sigma_3\otimes I_2.
\end{equation}
This is one valid Euclidean representation; physical operator identities are representation-independent.

\section{General d-Dimensional Construction}
The code builds gamma matrices recursively:
\begin{itemize}
  \item Start at $d=2$ with $(\sigma_1,\sigma_2)$.
  \item To extend from $2k$ to $2k+2$:
  \begin{align}
  \gamma'_\mu &= \sigma_1\otimes\gamma_\mu,\qquad \mu=0,\dots,2k-1,\\
  \gamma'_{2k} &= \sigma_2\otimes I,\qquad
  \gamma'_{2k+1} = \sigma_3\otimes I.
  \end{align}
  \item For odd $d=2k+1$, append
  \begin{equation}
  \gamma_{2k} = i^k\prod_{\mu=0}^{2k-1}\gamma_\mu.
  \end{equation}
\end{itemize}

\section{Wilson Dslash Convention}
In jaxQFT (aligned with Chroma Wilson dslash),
\begin{equation}
D' \psi(x) =
\sum_\mu \left[
U_\mu(x)\,(1-\mathrm{isign}\,\gamma_\mu)\,\psi(x+\hat\mu)
+
U^\dagger_\mu(x-\hat\mu)\,(1+\mathrm{isign}\,\gamma_\mu)\,\psi(x-\hat\mu)
\right].
\end{equation}
So for the undaggered branch (\texttt{isign=PLUS} in Chroma),
the forward hop uses $(1-\gamma_\mu)$ and backward uses $(1+\gamma_\mu)$.

Changing representation by $\gamma_\mu\to-\gamma_\mu$ swaps these projector labels.
This is convention-equivalent provided all formulas are changed consistently.

\end{document}
